% include file for the Gnus refcard and booklet
\def\progver{5.0}\def\refver{5.0} % program and refcard versions
\def\date{16 September 1995}
\def\author{Vladimir Alexiev $<$vladimir@cs.ualberta.ca$>$}
\raggedbottom\raggedright
\newlength{\logowidth}\setlength{\logowidth}{6.861in}
\newlength{\logoheight}\setlength{\logoheight}{7.013in}
\newlength{\keycolwidth}
\newenvironment{keys}[1]% #1 is the widest key
  {\nopagebreak%\noindent%
   \settowidth{\keycolwidth}{#1}%
   \addtolength{\keycolwidth}{\tabcolsep}%
   \addtolength{\keycolwidth}{-\columnwidth}%
   \begin{tabular}{@{}l@{\hspace{\tabcolsep}}p{-\keycolwidth}@{}}}%
  {\end{tabular}\\}
\catcode`\^=12 % allow ^ to be typed literally
\newcommand{\B}[1]{{\bf#1})}    % bold l)etter

\def\Title{
\begin{center}
{\bf\LARGE Gnus \progver\ Reference \Guide\\}
%{\normalsize \Guide\ version \refver}
\end{center}
}

\newcommand\Logo[1]{\centerline{
\makebox[\logoscale\logowidth][l]{\vbox to \logoscale\logoheight
{\vfill\special{psfile=gnuslogo.#1}}\vspace{-\baselineskip}}}}

\def\CopyRight{
\begin{center}
Copyright \copyright\ 1995 Free Software Foundation, Inc.\\*
Copyright \copyright\ 1995 \author.\\*
Created from the Gnus manual Copyright \copyright\ 1994 Lars Magne
Ingebrigtsen.\\*
and the Emacs Help Bindings feature (C-h b).\\*
Gnus logo copyright \copyright\ 1995 Luis Fernandes.\\*
\end{center}

Permission is granted to make and distribute copies of this reference
\guide{} provided the copyright notice and this permission are preserved on
all copies.  Please send corrections, additions and suggestions to the
above email address. \Guide{} last edited on \date.
}

\def\Notes{
\subsec{Notes}
{\samepage
Gnus is complex. Currently it has some 346 interactive (user-callable)
functions. Of these 279 are in the two major modes (Group and
Summary/Article). Many of these functions have more than one binding, some
have 3 or even 4 bindings. The total number of keybindings is 389. So in
order to save 40\% space, every function is listed only once on this
\guide, under the ``more logical'' binding. Alternative bindings are given
in parentheses in the beginning of the description.

Many Gnus commands are affected by the numeric prefix. Normally you enter a
prefix by holding the Meta key and typing a number, but in most Gnus modes
you don't need to use Meta since the digits are not self-inserting. The
prefixed behavior of commands is given in [brackets]. Often the prefix is
used to specify:

\quad [distance] How many objects to move the point over.

\quad [scope] How many objects to operate on (including the current one).

\quad [p/p] The ``Process/Prefix Convention'': If a prefix is given then it
determines how many objects to operate on. Else if there are some objects
marked with the process mark \#, these are operated on. Else only the
current object is affected.

\quad [level] A group subscribedness level. Only groups with a lower or
equal level will be affected by the operation. If no prefix is given,
`gnus-group-default-list-level' is used.  If
`gnus-group-use-permanent-levels', then a prefix to the `g' and `l'
commands will also set the default level.

\quad [score] An article score. If no prefix is given,
`gnus-summary-default-score' is used.
%Some functions were not yet documented at the time of creating this
%\guide and are clearly indicated as such.
\\*[\baselineskip]
\begin{keys}{C-c C-i}
C-c C-i & Go to the Gnus online {\bf info}.\\
C-c C-b & Send a Gnus {\bf bug} report.\\
\end{keys}
}}

\def\GroupLevels{
\subsec{Group Subscribedness Levels}
The table below assumes that you use the default Gnus levels.
Fill your user-specific levels in the blank cells.\\[1\baselineskip]

\begin{tabular}{|c|l|l|}
\hline
Level & Groups & Status \\
\hline
1 & mail groups   &              \\
2 & mail groups   &              \\
3 &               & subscribed   \\
4 &               &              \\
5 & default list level &         \\
\hline
6 &               & unsubscribed \\
7 &               &              \\
\hline
8 &               & zombies      \\
\hline
9 &               & killed       \\
\hline
\end{tabular}
}

\def\Marks{
\subsec{Mark Indication Characters}
{\samepage If a command directly sets a mark, it is shown in parentheses.\\*
\newlength{\markcolwidth}
\settowidth{\markcolwidth}{` '}% widest character
\addtolength{\markcolwidth}{4\tabcolsep}
\addtolength{\markcolwidth}{-\columnwidth}
\newlength{\markdblcolwidth}
\setlength{\markdblcolwidth}{\columnwidth}
\addtolength{\markdblcolwidth}{-2\tabcolsep}
\begin{tabular}{|c|p{-\markcolwidth}|}
\hline
\multicolumn{2}{|p{\markdblcolwidth}|}{{\bf ``Read'' Marks.}
  All these marks appear in the first column of the summary line, and so
  are mutually exclusive.}\\ 
\hline
` ' & (M-u, M SPC, M c) Not read.\\
!   & (!, M !, M t) Ticked (interesting).\\
?   & (?, M ?) Dormant (only followups are interesting).\\
C   & (C, S c) {\bf Canceled} (only for your own articles).\\
E   & (E, M e, M x) {\bf Expirable}. Only has effect in mail groups.\\
\hline\hline
\multicolumn{2}{|p{\markdblcolwidth}|}{The marks below mean that the article
  is read (killed, uninteresting), and have more or less the same effect.
  Some commands however explicitly differentiate between them (e.g.\ M
  M-C-r, adaptive scoring).}\\
\hline
r   & (d, M d, M r) Deleted (marked as {\bf read}).\\
C   & (M C; M C-c; M H; c, Z c; Z n; Z C) Killed by {\bf catch-up}.\\
O   & {\bf Old} (marked read in a previous session).\\
K   & (k, M k; C-k, M K) {\bf Killed}.\\
R   & {\bf Read} (viewed in actuality).\\
X   & Killed by a kill file.\\
Y   & Killed due to low score.\\
\hline\multicolumn{2}{c}{\vspace{1ex}}\\\hline
\multicolumn{2}{|p{\markdblcolwidth}|}{{\bf Other marks}}\\
\hline
\#  & (\#, M \#, M P p) Processable (will be affected by the next operation).\\
A   & {\bf Answered} (followed-up or replied).\\
+   & Over default score.\\
$-$ & Under default score.\\
=   & Has children (thread underneath it). Add `\%e' to
      `gnus-summary-line-format'.\\
\hline
\end{tabular}
}}

\def\GroupMode{
\sec{Group Mode}
\begin{keys}{C-c M-C-x}
RET     & (=) Select this group. [Prefix: how many (read) articles to fetch.
Positive: newest articles, negative: oldest ones.]\\
SPC     & Select this group and display the first unread article. [Same
prefix as above.]\\ 
?       & Give a very short help message.\\
$<$     & Go to the beginning of the Group buffer.\\
$>$     & Go to the end of the Group buffer.\\
,       & Jump to the lowest-level group with unread articles.\\
.       & Jump to the first group with unread articles.\\
^       & Enter the Server buffer mode.\\
a       & Post an {\bf article} to a group.\\
b       & Find {\bf bogus} groups and delete them.\\
c       & Mark all unticked articles in this group as read ({\bf catch-up}).
[p/p]\\ 
g       & Check the server for new articles ({\bf get}). [level]\\
j       & {\bf Jump} to a group.\\
m       & {\bf Mail} a message to someone.\\
n       & Go to the {\bf next} group with unread articles. [distance]\\
p       & (DEL) Go to the {\bf previous} group with unread articles.
[distance]\\ 
q       & {\bf Quit} Gnus.\\
r       & Read the init file ({\bf reset}).\\
s       & {\bf Save} the `.newsrc.eld' file (and `.newsrc' if
`gnus-save-newsrc-file').\\ 
z       & Suspend (kill all buffers of) Gnus.\\
B       & {\bf Browse} a foreign server.\\
C       & Mark all articles in this group as read ({\bf Catch-up}). [p/p]\\
F       & {\bf Find} new groups and process them.\\
N       & Go to the {\bf next} group. [distance]\\
P       & Go to the {\bf previous} group. [distance]\\
Q       & {\bf Quit} Gnus without saving any startup (.newsrc) files.\\
R       & {\bf Restart} Gnus.\\
V       & Display the Gnus {\bf version} number.\\
Z       & Clear the dribble buffer.\\
C-c C-d & Show the {\bf description} of this group. [Prefix: re-read it
from the server.]\\ 
C-c C-s & {\bf Sort} the groups by name, number of unread articles, or level
(depending on `gnus-group-sort-function').\\
C-c C-x & Run all expirable articles in this group through the {\bf expiry} 
process.\\
C-c M-C-x & Run all articles in all groups through the {\bf expiry} process.\\
C-x C-t & {\bf Transpose} two groups.\\
M-d     & {\bf Describe} ALL groups. [Prefix: re-read the description from the
server.]\\
M-f     & Fetch this group's {\bf FAQ} (using ange-ftp).\\
M-g     & Check the server for new articles in this group ({\bf get}). [p/p]\\
M-n     & Go to the {\bf previous} unread group on the same or lower level.
[distance]\\ 
M-p     & Go to the {\bf next} unread group on the same or lower level.
[distance]\\ 
\end{keys}
}

\def\GroupCommands{
\subsec{List Groups}
{\samepage
\begin{keys}{A m}
A a     & (C-c C-a) List all groups whose names match a regexp ({\bf
apropos}).\\ 
A d     & List all groups whose names or {\bf descriptions} match a regexp.\\ 
A k     & (C-c C-l) List all {\bf killed} groups.\\
A m     & List groups that {\bf match} a regexp and have unread articles.
[level]\\ 
A s     & (l) List {\bf subscribed} groups with unread articles. [level]\\
A u     & (L) List all groups (including {\bf unsubscribed}). [If no prefix
is given, level 7 is the default]\\ 
A z     & List the {\bf zombie} groups.\\
A M     & List groups that {\bf match} a regexp.\\
\end{keys}
}

\subsec{Create/Edit Foreign Groups}
{\samepage
The select methods are indicated in parentheses.\\*
\begin{keys}{G m}
G a     & Make the Gnus list {\bf archive} group. (nndir over ange-ftp)\\
G d     & Make a {\bf directory} group (every file must be a posting and files
must have numeric names). (nndir)\\
G e     & (M-e) {\bf Edit} this group's select method.\\
G f     & Make a group based on a {\bf file}. (nndoc)\\
G h     & Make the Gnus {\bf help} (documentation) group. (nndoc)\\
G k     & Make a {\bf kiboze} group. (nnkiboze)\\
G m     & {\bf Make} a new group.\\
G p     & Edit this group's {\bf parameters}.\\
G v     & Add this group to a {\bf virtual} group. [p/p]\\
G D     & Enter a {\bf directory} as a (temporary) group. (nneething without
recording articles read.)\\
G E     & {\bf Edit} this group's info (select method, articles read, etc).\\
G V     & Make a new empty {\bf virtual} group. (nnvirtual)\\
\end{keys}
You can also create mail-groups and read your mail with Gnus (very useful
if you are subscribed to any mailing lists), using one of the methods
nnmbox, nnbabyl, nnml, nnmh, or nnfolder. Read about it in the online info
(C-c C-i g Reading Mail RET).
}

%\subsubsec{Soup Commands}
%\begin{keys}{G s w}
%G s b   & gnus-group-brew-soup: not documented.\\
%G s p   & gnus-soup-pack-packet: not documented.\\
%G s r   & nnsoup-pack-replies: not documented.\\
%G s s   & gnus-soup-send-replies: not documented.\\
%G s w   & gnus-soup-save-areas: not documented.\\
%\end{keys}

\subsec{Mark Groups}
\begin{keys}{M m}
M m     & (\#) Set the process {\bf mark} on this group. [scope]\\
M u     & (M-\#) Remove the process mark from this group ({\bf unmark}).
[scope]\\ 
M w     & Mark all groups in the current region.\\
\end{keys}

\subsec{Unsubscribe, Kill and Yank Groups}
\begin{keys}{S w}
S k     & (C-k) {\bf Kill} this group.\\
S l     & Set the {\bf level} of this group. [p/p]\\
S s     & (U) Prompt for a group and toggle its {\bf subscription}.\\
S t     & (u) {\bf Toggle} subscription to this group. [p/p]\\
S w     & (C-w) Kill all groups in the region.\\
S y     & (C-y) {\bf Yank} the last killed group.\\
S z     & Kill all {\bf zombie} groups.\\
\end{keys}
}

\def\SummaryMode{
\sec{Summary Mode}  %{Summary and Article Modes}
\begin{keys}{SPC}
SPC     & (A SPC, A n) Select an article, scroll it one page, move to the
next one.\\ 
DEL     & (A DEL, A p, b) Scroll this article one page back. [distance]\\
RET     & Scroll this article one line forward. [distance]\\
=       & Expand the Summary window. [Prefix: shrink it to display the
Article window]\\
$<$     & (A $<$, A b) Scroll to the beginning of this article.\\
$>$     & (A $>$, A e) Scroll to the end of this article.\\
\&      & Execute a command on all articles matching a regexp.
[Prefix: move backwards.]\\
j       & (G g) Ask for an article number and then {\bf jump} to that summary
line.\\ 
C-t     & Toggle {\bf truncation} of summary lines.\\
M-\&    & Execute a command on all articles having the process mark.\\
M-k     & Edit this group's {\bf kill} file.\\
M-n     & (G M-n) Go to the {\bf next} summary line of an unread article.
[distance]\\ 
M-p     & (G M-p) Go to the {\bf previous} summary line of an unread article. 
[distance]\\ 
M-r     & Search through all previous articles for a regexp.\\
M-s     & {\bf Search} through all subsequent articles for a regexp.\\
M-K     & Edit the general {\bf kill} file.\\
\end{keys}
}

\def\SortSummary{
\subsec{Sort the Summary Buffer}
\begin{keys}{C-c C-s C-a}
C-c C-s C-a & Sort the summary by {\bf author}.\\
C-c C-s C-d & Sort the summary by {\bf date}.\\
C-c C-s C-i & Sort the summary by article score.\\
C-c C-s C-n & Sort the summary by article {\bf number}.\\
C-c C-s C-s & Sort the summary by {\bf subject}.\\
\end{keys}
}

\def\Asubmap{
\subsec{Article Buffer Commands}
\begin{keys}{A m}
A g     & (g) (Re)fetch this article ({\bf get}). [Prefix: just show the
article.]\\ 
A r     & (^, A ^) Go to the parent of this article (the {\bf References}
header).\\ 
M-^     & Fetch the article with a given Message-ID.\\
A s     & (s) Perform an i{\bf search} in the article buffer.\\
A D     & (C-d) Un{\bf digestify} this article into a separate group.\\
\end{keys}
}

\def\Bsubmap{
\subsec{Mail-Group Commands}
{\samepage
These commands (except `B c') are only valid in a mail group.\\*
\begin{keys}{B M-C-e}
B DEL   & {\bf Delete} the mail article from disk (!). [p/p]\\
B c     & {\bf Copy} this article from any group to a mail group. [p/p]\\
B e     & {\bf Expire} all expirable articles in this group. [p/p]\\
B i     & {\bf Import} a random file into this group.\\
B m     & {\bf Move} the article from one mail group to another. [p/p]\\
B q     & {\bf Query} where will the article go during fancy splitting\\
B r     & {\bf Respool} this mail article. [p/p]\\
B w     & (e) Edit this article.\\
B M-C-e & {\bf Expunge} (delete from disk) all expirable articles in this group
(!). [p/p]\\ 
\end{keys}
}}

\def\Gsubmap{
\subsec{Select Articles}
{\samepage
These commands select the target article. They do not understand the prefix.\\*
\begin{keys}{G C-n}
G b     & (,) Go to the {\bf best} article (the one with highest score).\\
G f     & (.) Go to the {\bf first} unread article.\\
G l     & (l) Go to the {\bf last} article read.\\
G n     & (n) Go to the {\bf next} unread article.\\
p       & Go to the {\bf previous} unread article.\\
G p     & {\bf Pop} an article off the summary history and go to it.\\
G N     & (N) Go to {\bf the} next article.\\
G P     & (P) Go to the {\bf previous} article.\\
G C-n   & (M-C-n) Go to the {\bf next} article with the same subject.\\
G C-p   & (M-C-p) Go to the {\bf previous} article with the same subject.\\
\end{keys}
}}

\def\Hsubmap{
\subsec{Help Commands}
\begin{keys}{H d}
H d     & (C-c C-d) {\bf Describe} this group. [Prefix: re-read the description
from the server.]\\
H f     & Try to fetch the {\bf FAQ} for this group using ange-ftp.\\
H h     & Give a very short {\bf help} message.\\
H i     & (C-c C-i) Go to the Gnus online {\bf info}.\\
H v     & Display the Gnus {\bf version} number.\\
\end{keys}
}

\def\Msubmap{
\subsec{Mark Articles}
\begin{keys}{M M-C-r}
d       & (M d, M r) Mark this article as read and move to the next one.
[scope]\\ 
D       & Mark this article as read and move to the previous one. [scope]\\
u       & (!, M !, M t) Tick this article (mark it as interesting) and move
to the next one. [scope]\\
U       & Tick this article and move to the previous one. [scope]\\ 
M-u     & (M SPC, M c) Clear all marks from this article and move to the next
one. [scope]\\ 
M-U     & Clear all marks from this article and move to the previous one.
[scope]\\ 
M ?     & (?) Mark this article as dormant (only followups are
interesting). [scope]\\ 
M b     & Set a {\bf bookmark} in this article.\\
M e     & (E, M x) Mark this article as {\bf expirable}. [scope]\\
M k     & (k) {\bf Kill} all articles with the same subject then select the
next one.\\ 
M B     & Remove the {\bf bookmark} from this article.\\
M C     & {\bf Catch-up} the articles that are not ticked.\\
M D     & Show all {\bf dormant} articles (normally they are hidden unless they
have any followups).\\
M H     & Catch-up (mark read) this group to point ({\bf here}).\\
M K     & (C-k) {\bf Kill} all articles with the same subject as this one.\\
C-w     & Mark all articles between point and mark as read.\\
M S     & (C-c M-C-s) {\bf Show} all expunged articles.\\
M C-c   & {\bf Catch-up} all articles in this group.\\
M M-r   & (x) Expunge all {\bf read} articles from this group.\\
M M-D   & Hide all {\bf dormant} articles.\\
M M-C-r & Expunge all articles having a given mark.\\
\end{keys}

\subsubsec{Mark Based on Score}
\begin{keys}{M s m}
M V c   & {\bf Clear} all marks from all high-scored articles. [score]\\
M V k   & {\bf Kill} all low-scored articles. [score]\\
M V m   & Mark all high-scored articles with a given {\bf mark}. [score]\\
M V u   & Mark all high-scored articles as interesting (tick them). [score]\\
\end{keys}

\subsubsec{The Process Mark}
{\samepage 
These commands set and remove the process mark \#. You only need to use
it if the set of articles you want to operate on is non-contiguous. Else
use a numeric prefix.\\*
\begin{keys}{M P R}
M P a   & Mark {\bf all} articles (in series order).\\
M P p   & (\#, M \#) Mark this article.\\
M P r   & Mark all articles in the {\bf region}.\\
M P s   & Mark all articles in the current {\bf series}.\\
M P t   & Mark all articles in this (sub){\bf thread}.\\
M P u   & (M-\#, M M-\#) {\bf Unmark} this article.\\
M P R   & Mark all articles matching a {\bf regexp}.\\
M P S   & Mark all {\bf series} that already contain a marked article.\\
M P U   & {\bf Unmark} all articles.\\
\end{keys}
}}

\def\Osubmap{
\subsec{Output Articles}
\begin{keys}{O m}
O f     & Save this article in plain {\bf file} format. [p/p]\\
O h     & Save this article in {\bf mh} folder format. [p/p]\\
O m     & Save this article in {\bf mail} format. [p/p]\\
O o     & (o, C-o) Save this article using the default article saver. [p/p]\\
O p     & ($\mid$) Pipe this article to a shell command. [p/p]\\
O r     & Save this article in {\bf rmail} format. [p/p]\\
O v     & Save this article in {\bf vm} format. [p/p]\\
\end{keys}
}

\def\Ssubmap{
\subsec{Post, Followup, Reply, Forward, Cancel}
{\samepage
These commands put you in a separate post or mail buffer. After
editing the article, send it by pressing C-c C-c.  If you are in a
foreign group and want to post the article using the foreign server, give
a prefix to C-c C-c.\\* 
\begin{keys}{S O m}
S b     & {\bf Both} post a followup to this article, and send a reply.\\
S c     & (C) {\bf Cancel} this article (only works if it is your own).\\
S f     & (f) Post a {\bf followup} to this article.\\
S m     & (m) Send {\bf a} mail to some other person.\\
S o m   & (C-c C-f) Forward this article by {\bf mail} to a person.\\
S o p   & Forward this article as a {\bf post} to a newsgroup.\\
S p     & (a) {\bf Post} an article to this group.\\
S r     & (r) Mail a {\bf reply} to the author of this article.\\
S s     & {\bf Supersede} this article with a new one (only for own
articles).\\ 
S u     & {\bf Uuencode} a file and post it as a series.\\
S B     & {\bf Both} post a followup, send a reply, and include the
original. [p/p]\\ 
S F     & (F) Post a {\bf followup} and include the original. [p/p]\\
S O m   & Digest these series and forward by {\bf mail}. [p/p]\\
S O p   & Digest these series and forward as a {\bf post} to a newsgroup.
[p/p]\\ 
S R     & (R) Mail a {\bf reply} and include the original. [p/p]\\
\end{keys}
If you want to cancel or supersede an article you just posted (before it
has appeared on the server), go to the *post-news* buffer, change
`Message-ID' to `Cancel' or `Supersedes' and send it again with C-c C-c.
}}

\def\Tsubmap{
\subsec{Thread Commands}
\begin{keys}{T \#}
T \#    & Mark this thread with the process mark.\\
T d     & Move to the next article in this thread ({\bf down}). [distance]\\
T h     & {\bf Hide} this (sub)thread.\\
T i     & {\bf Increase} the score of this thread.\\
T k     & (M-C-k) {\bf Kill} the current (sub)thread. [Negative prefix:
tick it, positive prefix: unmark it.]\\
T l     & (M-C-l) {\bf Lower} the score of this thread.\\
T n     & (M-C-f) Go to the {\bf next} thread. [distance]\\
T p     & (M-C-b) Go to the {\bf previous} thread. [distance]\\
T s     & {\bf Show} the thread hidden under this article.\\
T u     & Move to the previous article in this thread ({\bf up}). [distance]\\
T H     & {\bf Hide} all threads.\\
T S     & {\bf Show} all hidden threads.\\
T T     & (M-C-t) {\bf Toggle} threading.\\
\end{keys}
}

\def\Vsubmap{
\subsec{Score (Value) Commands}
{\samepage
Read about Adaptive Scoring in the online info.\\*
\begin{keys}{\bf A p m l}
V a     & {\bf Add} a new score entry, specifying all elements.\\
V c     & Specify a new score file as {\bf current}.\\
V e     & {\bf Edit} the current score alist.\\
V f     & Edit a score {\bf file} and make it the current one.\\
V m     & {\bf Mark} all articles below a given score as read.\\
V s     & Set the {\bf score} of this article.\\
V t     & Display all score rules applied to this article ({\bf track}).\\
V x     & {\bf Expunge} all low-scored articles. [score]\\
V C     & {\bf Customize} the current score file through a user-friendly
interface.\\ 
V S     & Display the {\bf score} of this article.\\
\bf A p m l& Make a scoring entry based on this article.\\
\end{keys}

The four letters stand for:\\*
\quad \B{A}ction: I)ncrease, L)ower;\\*
\quad \B{p}art: a)utor (from), s)ubject, x)refs (cross-posting), d)ate, l)ines,
message-i)d, t)references (parent), f)ollowup, b)ody, h)ead (all headers);\\*
\quad \B{m}atch type:\\*
\qquad string: s)ubstring, e)xact, r)egexp, f)uzzy,\\*
\qquad date: b)efore, a)t, n)this,\\*
\qquad number: $<$, =, $>$;\\*
\quad \B{l}ifetime: t)emporary, p)ermanent, i)mmediate.

If you type the second letter in uppercase, the remaining two are assumed
to be s)ubstring and t)emporary. 
If you type the third letter in uppercase, the last one is assumed to be 
t)emporary.

\quad Extra keys for manual editing of a score file:\\*
\begin{keys}{C-c C-c}
C-c C-c & Finish editing the score file.\\
C-c C-d & Insert the current {\bf date} as number of days.\\
\end{keys}
}}

\def\Wsubmap{
\subsec{Wash the Article Buffer}
\begin{keys}{W C-c}
W b     & Make Message-IDs and URLs in the article to mouse-clickable {\bf
  buttons}.\\  
W c     & Remove extra {\bf CRs} (^M) from the article.\\
W f     & Look for and display any X-{\bf Face} headers.\\
W l     & (w) Remove page breaks ({\bf^L}) from the article.\\
W m     & Toggle {\bf MIME} processing.\\
W o     & Treat {\bf overstrike} or underline (^H\_) in the article.\\
W q     & Treat {\bf quoted}-printable in the article.\\
W r     & (C-c C-r) Do a Caesar {\bf rotate} (rot13) on the article.\\
W t     & (t) {\bf Toggle} the displaying of all headers.\\
v       & Toggle permanent {\bf verbose} displaying of all headers.\\
W w     & Do word {\bf wrap} in the article.\\
W T e   & Convert the article timestamp to time {\bf elapsed} since sent.\\
W T l   & Convert the article timestamp to the {\bf local} timezone.\\
W T u   & (W T z) Convert the article timestamp to {\bf UTC} ({\bf Zulu},
GMT).\\ 
\end{keys}

\subsubsec{Hide/Highlight Parts of the Article}
\begin{keys}{W W C-c}
W W a   & Hide {\bf all} unwanted parts. Calls W W h, W W s, W W C-c.\\
W W c   & Hide article {\bf citation}.\\
W W h   & Hide article {\bf headers}.\\
W W s   & Hide article {\bf signature}.\\
W W C-c & Hide article {\bf citation} using a more intelligent algorithm.\\
%\end{keys}
%
%\subsubsec{Highlight Parts of the Article}
%\begin{keys}{W H A}
W H a   & Highlight {\bf all} parts. Calls W b, W H c, W H h, W H s.\\
W H c   & Highlight article {\bf citation}.\\
W H h   & Highlight article {\bf headers}.\\
W H s   & Highlight article {\bf signature}.\\
\end{keys}
}

\def\Xsubmap{
\subsec{Extract Series (Uudecode etc)}
{\samepage
Gnus recognizes if the current article is part of a series (multipart
posting whose parts are identified by numbers in their subjects, e.g.{}
1/10\dots10/10) and processes the series accordingly. You can mark and
process more than one series at a time. If the posting contains any
archives, they are expanded and gathered in a new group.\\*
\begin{keys}{X p}
X b     & Un-{\bf binhex} these series. [p/p]\\
X o     & Simply {\bf output} these series (no decoding). [p/p]\\ 
X p     & Unpack these {\bf postscript} series. [p/p]\\
X s     & Un-{\bf shar} these series. [p/p]\\
X u     & {\bf Uudecode} these series. [p/p]\\
\end{keys}

Each one of these commands has four variants:\\*
\begin{keys}{X v \bf Z}
X   \bf z & Decode these series. [p/p]\\
X   \bf Z & Decode and save these series. [p/p]\\
X v \bf z & Decode and view these series. [p/p]\\
X v \bf Z & Decode, save and view these series. [p/p]\\
\end{keys}
where {\bf z} or {\bf Z} identifies the decoding method (b, o, p, s, u).

An alternative binding for the most-often used of these commands is\\*
\begin{keys}{C-c C-v C-v}
C-c C-v C-v & (X v u) Uudecode and view these series. [p/p]\\
\end{keys}
}}

\def\Zsubmap{
\subsec{Exit the Current Group}
\begin{keys}{Z G}
Z c     & (c) Mark all unticked articles as read ({\bf catch-up}) and exit.\\
Z n     & Mark all articles as read and go to the {\bf next} group.\\
Z C     & Mark all articles as read ({\bf catch-up}) and exit.\\
Z E     & (Q) {\bf Exit} without updating the group information.\\
Z G     & (M-g) Check for new articles in this group ({\bf get}).\\
Z N     & Exit and go to {\bf the} next group.\\
Z P     & Exit and go to the {\bf previous} group.\\
Z R     & Exit this group, and then enter it again ({\bf reenter}).
[Prefix: select all articles, read and unread.]\\
Z Z     & (q, Z Q) Exit this group.\\
\end{keys}
}

\def\ArticleMode{
\sec{Article Mode}
{\samepage
% All keys for Summary mode also work in Article mode.
The normal navigation keys work in Article mode.
Some additional keys are:\\*
\begin{keys}{C-c C-m}
RET     & (middle mouse button) Activate the button at point to follow
an URL or Message-ID.\\
TAB     & Move the point to the next button.\\
h       & (s) Go to the {\bf header} line of the article in the {\bf
summary} buffer.\\ 
C-c ^   & Get the article with the Message-ID near point.\\
C-c C-m & {\bf Mail} reply to the address near point (prefix: include the
original).\\ 
\end{keys}
}}

\def\ServerMode{
\sec{Server Mode}
{\samepage
To enter this mode, press `^' while in Group mode.\\*
\begin{keys}{SPC}
SPC     & (RET) Browse this server.\\
a       & {\bf Add} a new server.\\
c       & {\bf Copy} this server.\\
e       & {\bf Edit} a server.\\
k       & {\bf Kill} this server. [scope]\\
l       & {\bf List} all servers.\\
q       & Return to the group buffer ({\bf quit}).\\
y       & {\bf Yank} the previously killed server.\\
\end{keys}
}}

\def\BrowseServer{
\sec{Browse Server Mode}
{\samepage
To enter this mode, press `B' while in Group mode.\\*
\begin{keys}{RET}
RET     & Enter the current group.\\
SPC     & Enter the current group and display the first article.\\
?       & Give a very short help message.\\
n       & Go to the {\bf next} group. [distance]\\
p       & Go to the {\bf previous} group. [distance]\\
q       & (l) {\bf Quit} browse mode.\\
u       & Subscribe to the current group. [scope]\\
\end{keys}
}}
