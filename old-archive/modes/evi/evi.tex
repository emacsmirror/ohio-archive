% -*-texinfo-*-

\input texinfo

@setfilename evi.info
@settitle Evi - Enhanced VI for GNU Emacs
@setchapternewpage odd

@iftex
@finalout
@end iftex

@ifinfo
This file documents Evi - a vi emulator for GNU Emacs

Copyright @copyright{} 1992, 1993  Jeffrey R. Lewis

Permission is granted to make and distribute verbatim copies of
this manual provided the copyright notice and this permission notice
are preserved on all copies.

@ignore
Permission is granted to process this file through Tex and print the
results, provided the printed document carries copying permission
notice identical to this one except for the removal of this paragraph
(this paragraph not being relevant to the printed manual).
@end ignore

Permission is granted to copy and distribute modified versions of this
manual under the conditions for verbatim copying, provided that the entire
resulting derived work is distributed under the terms of a permission
notice identical to this one.

Permission is granted to copy and distribute translations of this manual
into another language, under the same conditions as for modified versions.
@end ifinfo

@titlepage
@title Evi - Enhanced vi for GNU Emacs
@subtitle June 1993, Evi Version 0.99.8

@author Jeffrey R. Lewis

@page
@vskip 0pt plus 1filll

Copyright @copyright{} 1992, 1993  Jeffrey R. Lewis

Permission is granted to make and distribute verbatim copies of
this manual provided the copyright notice and this permission notice
are preserved on all copies.

@ignore
Permission is granted to process this file through Tex and print the
results, provided the printed document carries copying permission
notice identical to this one except for the removal of this paragraph
(this paragraph not being relevant to the printed manual).
@end ignore

Permission is granted to copy and distribute modified versions of this
manual under the conditions for verbatim copying, provided that the entire
resulting derived work is distributed under the terms of a permission
notice identical to this one.

Permission is granted to copy and distribute translations of this manual
into another language, under the same conditions as for modified versions.
@end titlepage

@node Top,Overview,(DIR),(DIR)

@menu
* Overview::			What is Evi?
* Setting up Evi::		How to start using Evi
* File and buffer management::	How vi file and buffer commands
				work under Evi, plus Evi extensions
* Window management::		How vi window management commands
				work under Evi, plus Evi extensions
* Emacs commands::		How to access emacs commands from Evi
* Emacs editing modes::		Taking advantage of emacs editing modes
* Customizing::			How to customize Evi
* Enhancements::		Various Evi enhancements to vi
* New operators::		New commands that operate over regions
* Differences::			Differences between Evi and vi
* Ex commands::			Supported ex commands and variable settings
* Note to vip users::		Differences between Evi and vip
* Bugs::			What to do if things go wrong
* Mailing list::		How to join and submit to the Evi mailing list
* Getting a copy of Evi::	How and where to get more recent copies of Evi
* Credits::			Who contributed to Evi
@end menu

@node Overview,Setting up Evi,Top,Top

Evi is a vi emulator aimed at those who are either just plain accustomed
to vi, or who just simply happen to like its style of editing better
than emacs' default.  Evi's first goal is vi compatibility.  Its second
goal is to be an extension of vi, taking advantage of features supplied
by the emacs environment, without simply becoming emacs with vi'ish key
bindings.

Ideally, you shouldn't need any special manual to start using Evi, other
than your vi manual.  By default evi is set-up to emulate vi as closely
as possible, including using your EXINIT environment variable and/or
.exrc startup files.  Of course, you'll need to know how to get started
with Evi, and you'll probably want to take advantage of some of Evi's
extensions, and be aware of what's different between vi and Evi.  That's
what this manual covers.

A note on conventions used in this manual: @kbd{M-x} means the escape
key followed by the character x (where x is any character), @kbd{C-x}
means control-x, @key{RET} means the return key,
@key{ESC} means the escape key. @refill

@node Setting up Evi,File and buffer management,Overview,Top
@section Setting up Evi

To just test Evi out, type:
@example
M-x load-file @key{RET} evi.el @key{RET}
M-x evi @key{RET}
@end example
(don't type the spaces) You will now be in the emulator.

Since evi sources your @file{.exrc} and/or @code{EXINIT}, it is possible
that evi may have problems due to some command or syntax that it doesn't
know about.  If this is the case, and you wish to simply suppress sourcing
of @file{.exrc} and @code{EXINIT}, place the
following in your @file{.emacs}: @refill
@example
(setq evi-supress-ex-startup t)
@end example

Then, send me a note with the contents of your @file{.exrc} and @code{EXINIT}
so we can fix the problem.
Actually, there shouldn't be much problem with this as
every command that makes much sense in a @file{.exrc} is supported in evi,
and unsupported options are silently ignored.
If you wish to find out what options evi is ignoring, put the following
in your @file{.evirc}: @refill
@example
(setq evi-report-unsupported-options t)
@end example

If you decide to continue using Evi, I would recommend first you byte
compile it to speed things up, using the following shell command:
@example
emacs -batch -f batch-byte-compile evi.el
@end example
Next, if you want to use Evi all the time, put the following at the end
of your @file{.emacs} file:
@example
(load "<wherever-evi-is>/evi")
(setq term-setup-hook 'evi)
@end example
this will make emacs go into Evi every time you fire it up.  Of course,
you may wish to have Evi properly installed for all to use - consult
your local emacs guru.

@node File and buffer management,Window management,Setting up Evi,Top
@section File and buffer management

Vi's file management commands have long tried to mimic having multiple
buffers, and as such, work well enough to use in an emacs setting.  They
of course have to take on slightly different meanings, since it makes
little sense to emulate the limitations of the vi/ex model that
presumably you are trying to avoid by using Evi!

@table @code
@item :e
Edit a file in the current window.  With no argument, brings
in a new copy of the file (useful if it has been subsequently
modified on disk).  @code{:e!} will override any complaints about
the current buffer being modified, and discards all
modifications.  With a filename argument, edits that file
in the current window (using the copy already in the editor
if it was previously read in).  There is no difference
between @code{:e! filename} and @code{:e filename}, because in Evi we
don't need to worry about the disposition of the current
file before editing the next one.  Use @code{:e#} as a shorthand for
editing the most recently accessed buffer not visible in a window. @refill
@item :E
Same as @code{:e}, but edits the file in another window, creating
that window if necessary.  If used with no filename, this command splits the
current buffer into two windows. @refill
@item :n
Switch to the next file in buffer list that's not currently
displayed.  Rotates the current file to the end of the buffer
list, so repeated use of @code{:n} will cycle thru all buffers. @refill
@item :N
Same as @code{:n}, but switches to another window, or creates another
window and puts the next file into it. @refill
@item :b
This is an extension.  This command switches the current
window to the specified buffer, e.g. @code{:b foo} would switch
to the buffer named @code{foo} (if it exists).  By default @code{:b}
switches to the next buffer not displayed.  @code{:b!} will create
the buffer if it doesn't exist. @refill
@item :B
Analogous to :E and :N.
@item :k
Another extension.  This command kills the named buffer.  If given
no argument, kills the current buffer.  @code{:k!} will kill the buffer
even if it is modified. @refill
@item :K
Like @code{:k}, but also kills all windows associated with the buffer. @refill
@item :wk
Writes the current buffer, and kills it.
@item :W
Another extension.  This command writes all modified buffers, querying
the user before each write.  @code{:W!} will
write all buffers with no questions asked. @refill
@item :Wq
As above, then exits emacs.  See @kbd{ZZ} below.
@end table

The ex commands which accept filenames as arguments can be file
completed using space or tab.  Similarly,
@code{:b} and @code{:k} will buffer-name complete.
One thing you might find handy is
@code{:b @key{TAB}} to see a list of buffers.
Use @kbd{C-c} or backspace to escape this. @refill

@node Window management,Emacs commands,File and buffer management,Top
@section Window management

Vi had commands for multiple file management, but it never tried to
pretend it handled multiple windows.  However, it did have the @kbd{z}
command for simple window management, like making the window bigger or
smaller.  In Evi, the @kbd{z} command is extended to handle most Emacs
window management needs.  Additional suggestions are welcome.

@table @kbd
@item z0=
Delete the current window from the screen.
@item z1=
Make the current window fill the screen.
@item z2=
Split the current window in two vertically.
@item z0|
Same as @kbd{z0=}.
@item z1|
Same as @kbd{z1=}.
@item z2|
Split the current window in two horizontally.
@item z<num>+
@item z<num>-
These two let you adjust the size of the current window by <num>.
Use z<num>. to adjust the window size absolutely.
@item zf
Go to the next window (forward).
@item zn
Same as @kbd{zf}.
@item zb
Go to the previous window (backward).
@item zp
Same as @kbd{zb}.
@item zH
@item zM
@item zL
These are aliases for @kbd{z@key{RET}}, @kbd{z.}, and @kbd{z-} and correspond
to the arguments to the mark command. @refill
@end table

@node Emacs commands,Emacs editing modes,Window management,Top
@section Accessing emacs commands

By default, no emacs commands are accessible from Evi.  You can enable
most emacs commands by @code{:set enable-emacs-commands}.  In particular,
this enables all emacs commands whose key sequence starts with any of:
@example
@kbd{C-a} @kbd{C-k} @kbd{C-o} @kbd{C-q} @kbd{C-s} @kbd{C-t}
@kbd{C-v} @kbd{C-w} @kbd{C-x} @kbd{C-\} @key{ESC}
@end example
If you wish to disable an evi command in favor of an emacs command, simply
bind the appropriate key to nil:
@example
:bind \C-y nil
@end example
This prevents Evi from handling the key, and with @code{enable-emacs-commands}
set, allows emacs to handle it.

In other modes, however, all keys are bound, so if you wish to access emacs
commands in, for example, insert mode, read on.  To access @kbd{C-x} prefix
commands in the various input modes, use the following:
@example
:bind! \C-x evi-emacs-command
@end example
Accessing @key{ESC} prefix commands in input modes is of course problematic
because @key{ESC} from any of these modes means to exit the mode.  To get
around this, I suggest using another key as your `meta prefix' in evi.
A good choice is @kbd{C-a}.  To effect this, the the @code{meta-prefix}
variable:
@example
:set meta-prefix \C-a
@end example

The @kbd{C-c} prefix commands are also problematic because @kbd{C-c} is vi's
interrupt character.  This is perhaps most useful in insert mode, and you
might like to:
@example
:bind! \C-c 'evi-emacs-command
@end example
Although you could also bind it in command mode:
@example
:bind \C-c 'evi-emacs-command
@end example
@kbd{C-c} will still function as an interrupt character at all other places
(including in the middle of a command, where it's most useful).
Note that setting evi-insert-mode-local-bindings (described in the next
section) will have the same effect because all @kbd{C-c} commands are local
bindings.

@node Emacs editing modes,Customizing,Emacs commands,Top
@section Taking advantage of emacs editing modes

A number of emacs editing modes have handy local key bindings other than
Meta, @kbd{C-x} and @kbd{C-c} prefix bindings.
For example, in C mode, @key{RET} does
intelligent indenting, and @kbd{@}} is bound to a command which automatically
exdents.  By default, these aren't accessible of course, but you can
have Evi enable these local bindings in insert mode by setting: @refill
@example
(setq evi-insert-mode-local-bindings t)
@end example
As current policy, however, Evi will not allow local mode bindings to
override @key{TAB}, @key{BS}, or @key{DEL}, as well as, for obvious reasons, @key{ESC}.
@key{ESC} prefix commands, however, can be accessed as described in the
previous section.

@node Customizing,Enhancements,Emacs editing modes,Top
@section Customizing Evi

Like vi, Evi will source your @file{.exrc} or @file{~/.exrc} file,
and/or your @code{EXINIT}
environment variable.  If your startup runs into problems, let me know -
you shouldn't have to change your vi initialization stuff to make Evi
happy. @refill

If you wish to use some Evi extensions in your startup, but still need
to use vi, place these in @file{.exrc.evi}, @file{~/.exrc.evi}
or @code{EVIINIT} so that vi won't gag on them. @refill

Emacs lisp startup code for evi, such at that suggested in the previous
sections, can be placed in either @file{.evirc} or @file{~/.evirc}.

And you can, of course, hack away at the the Evi source code if you want
something not easily addressed by the above methods.  If you feel what
you've done would be generally useful, please email it to me, or post it.

One particular customization, not covered elsewhere, is how Evi handles
the current directory.  By default it behaves like vi - you have one
global current directory, which you change using @code{:cd} (also see
@code{:pushd}, and friends described below).  Alternately, you may like the
emacs behaviour better, which is that each buffer has its own idea of
the current directory, and by default that directory is the directory
that the file for that buffer resides in.  In this mode, you can also
change the current directory using @code{:cd}, but that will only affect the
current buffer.  To get this behaviour, place the following in your
@file{.evirc}:
@example
(setq evi-global-directory nil)
@end example

Another customization you might like to make is to alter the behaviour
of @kbd{ZZ}.  By default it is bound to @code{:Wq!}, which quietly writes all
modified files and exits.  If, however, you would like to be asked about
each modified buffer before it is saved in order to avoid accidently
saving a file you didn't want saved, map @kbd{ZZ} to @code{:Wq}:
@example
map ZZ :Wq\n
@end example

@node Enhancements,New operators,Customizing,Top
@section Enhancements

@menu
* Command line editing::
* Edit repeat command::
* Ex input escapes::
* Extended undo::
* Ex command completion::
* Word definition::
* Sentence and larger motions::
* Directory commands::
* Background shell commands::
* Mail commands::
* Unnamed register::
* New command counts::
* Rectangle edits and arbitrary regions::
* Shell window::
* Extended marks::
* Register enhancements::
* Language/mode specific editing::
@end menu

@node Command line editing,Edit repeat command,,Enhancements
@subsection Command line editing

You can edit the command line of most commands that use the bottom line for
input, such as @kbd{:} and @kbd{/}.  The way this works is that when you
issue a command such as @kbd{:}, you are put on the command line in insert
mode, editing a special command buffer.  Thus, to go back to the beginning
of the line to fix a typo, just hit @key{ESC} to put you in command mode,
and make the change.  A @key{RET} in either insert or command mode will
terminate command line entry.  Each different command that uses the command
will also have associated with it a command history.  To peruse this, just
use the @kbd{j} and @kbd{k} commands - or any other motion command for that
matter, such as @kbd{/}.

However, since this feature changes the behaviour of @key{ESC} on the command
line (in vi it is synonymous with @key{RET}), this feature is disabled by
default.  To enable it, do a @code{:set command-line-editing}.

@node Edit repeat command,Ex input escapes,Command line editing,Enhancements
@subsection Edit repeat command

@kbd{_} is a new version of the repeat command @kbd{.}
that prompts you with the
keystrokes to repeat, allowing you to edit them before executing.  This
is particularly useful for the abovementioned complex operators.  If you
don't wish to re-execute the command, just hit @kbd{C-c}. @refill

@node Ex input escapes,Extended undo,Edit repeat command,Enhancements
@subsection Ex input escapes

If you put
@example
(setq ex-input-escapes t)
@end example
in your @file{.evirc}, then
Ex (@kbd{:}) commands will accept the
following escapes: @code{\e} for @key{ESC},
@code{\n} for newline, @code{\r} for @key{RET},
and @code{\C-x} for control-x (for all valid control characters).
@code{\} otherwise works like @kbd{C-v}.  Thus: @refill
@example
map \  \C-f
@end example
would make the space character be page forward, and
@example
map g \|
@end example
would make @kbd{g} be goto-column.  Note that @kbd{|} is normally a command
separator and thus must be escaped.

@node Extended undo,Ex command completion,Ex input escapes,Enhancements
@subsection Extended undo

The command @kbd{[u} continues the previous undo, by undoing one more change
previous to the last change undone.  Thus, a long enough sequence of
@kbd{[u}s will take you back to the unmodified state.  If you went back too
far, a @kbd{u} will reverse this process, and subsequent @kbd{[u}s will
move forward through the changes.
Note that vip does this using repeat command (@kbd{.}); however,
that conflicts with the meaning of @kbd{u.} in vi, which is: `undo, then do
again'.  This is quite handy for reapplying a change that you initially
did in the wrong place, so evi leaves that meaning alone, and
defines a new command for `extended undo'.
Personally, I use `extended undo' more than `undo line', so I swap
the two definitions in my .evirc: @refill
@example
(evi-define-key '(vi) "[u" 'evi-undo-line)
(evi-define-key '(vi) "U" 'evi-undo-more)
@end example

@node Ex command completion,Word definition,Extended undo,Enhancements
@subsection Ex command completion

In Ex (@kbd{:}) commands, you can use the @key{TAB} to perform completion
on the following: commands, variable names (:set), filenames, buffernames,
maps, or abbreviations.  Which completion to perform is determined solely
by where you are in the partial command you are typing.  For example:
@example
:set er@key{TAB}
@end example
would complete to:
@example
:set errorbells
@end example
(leaving the cursor after the @code{s} in @code{errorbells}.

@node Word definition,Sentence and larger motions,Ex command completion,Enhancements
@subsection Word definition

You can define exactly what Evi treats as words for the w, b, e, W, B
and E commands.  They are defined by setting either or both of the new
options @code{word} (for w, b and e) or
@code{Word} (for W, B and E) to a regular
expression describing what words look like.  For example, here's a
definition of words that only considers alphanumeric words: @refill
@example
set word=[a-zA-Z0-9]+
@end example
Contrast this with the default definition:
@example
[a-zA-Z0-9_]+\\\|[^a-zA-Z0-9_\ \t\n]+\\\|^[\ \t]*\n
@end example
See the emacs documentation on regular expressions for details.

@node Sentence and larger motions,Directory commands,Word definition,Enhancements
@subsection Sentence and larger motions

My interpretation of sentence, paragraph, and section motion differs
somewhat from vi's in that they behave more analogously to how word
motion behaves - e.g. a forward paragraph takes you to the beginning of
the next paragraph - not the blank line after the previous paragraph.
However, when doing a delete using one of these motions, unless you are
at the beginning of the sentence, paragraph or section, the delete will
only happen to the end of the sentence, paragraph or section, not to the
beginning of the next.  I find this *much* more useful than the vi
behaviour - if you disagree, please let me know.

@node Directory commands,Background shell commands,Sentence and larger motions,Enhancements
@subsection Directory commands

@code{:pushd}, @code{:popd} and @code{:dirs} commands exist,
similar to those found in
csh and bash.  Note these only make sense in conjunction with
evi-global-directory = t (which is the default). @refill

@node Background shell commands,Mail commands,Directory commands,Enhancements
@subsection Background shell commands

The @code{:!} command now takes an optional @code{&} (as in @code{:!&}),
which causes
the shell command to be executed asynchronously, with the output going
to the window `*Shell Command Output*'. @refill

@node Mail commands,Unnamed register,Background shell commands,Enhancements
@subsection Mail commands

The @code{:mail} command puts you in a buffer in which you can compose
a mail message.  The top of the buffer contains message headers which you
may edit, and you are placed at the beginning of the message section in
insert mode.  The @code{:mail} command takes an optional argument consisting
of the list of recipients, which is then placed in the header for further
editing if necessary.  When you are ready to send the message, use the
@code{:send} command.  This sends the message, but leaves you in the mail
buffer for further editing and sending.  Use the @code{:kill} command to
delete the buffer if you are done with it.  Alternately, @code{send!} sends
the message and kills the buffer. @refill

@node Unnamed register,New command counts,Mail commands,Enhancements
@subsection Unnamed register

The unnamed register (where deleted text goes) is preserved across
excursions into insert mode, etc.  This means you can delete something,
insert something, then `put' the deleted text.  In vi, for no apparent
reason, you can't do this, even though insert mode doesn't use the
unnamed register.

The unnamed register is also preserved between buffers, so you can yank
text in one buffer and put it into another.

@node New command counts,Rectangle edits and arbitrary regions,Unnamed register,Enhancements
@subsection New command counts

Several commands that didn't take counts in vi take counts in Evi.  @kbd{p}
and @kbd{P} take a prefix count and will put the text that many times,
regardless of the size of the text - vi will apparently only do the
prefix count for less than line sized text.  @kbd{/} takes a prefix count to
find the nth occurence of a string.  @kbd{D} takes a count (I could never
figure out why it didn't, since @kbd{C} takes a count).

@node Rectangle edits and arbitrary regions,Shell window,New command counts,Enhancements
@subsection Rectangle edits and arbitrary regions

In vi, you do most of your editing based on regions defined by vi's motion
commands.  For example, @kbd{dw} deletes the region starting at the current
cursor location and extending up to where the cursor would be if you'd typed
@kbd{w}.  Those edit commands that don't explicitly use a motion command are
just shorthand for ones that do: e.g. @kbd{x} is shorthand for @kbd{dl}.

However, this doesn't give you a convenient way of operating on arbitrary
rectangular regions, or regions that are inconvenient to describe by a
single motion command.  Evi allows you to operate on such regions by a
special form of the @kbd{m} (mark) command, and several new commands that
are only understood when they are operands to edit operators (such as @kbd{c},
@kbd{d}, @kbd{>}, etc).

The basic idea is that you will define a region by two points.  For a rectangle
region, for example, you will define two opposite corners.  You define the
first point by using @kbd{m.}.  Then you move to the second point and
execute the operator you want, specifying the motion to be one of: @kbd{r} for
rectangle, @kbd{R} for rows, @kbd{C} for columns, and @kbd{a} for arbitrary
(meaning everything between the marked point and where the cursor currently
is).
@example
m.3j5wda
@end example
would delete the text from where the cursor started to 3 lines down and
5 words over.  @kbd{R} is often handy for operating on large arbitrary
sections of text, for example say you needed to shift some text that ran
on for several pages and you weren't sure just how long it was at the
start:
@example
m.@kbd{C-f}@kbd{C-f}jjj>R
@end example

@node Shell window,Extended marks,Rectangle edits and arbitrary regions,Enhancements
@subsection Shell window

@code{:shell} starts up an emacs shell in the current window (instead of
suspending emacs, and starting a subshell).  The shell to run comes from
the vi variable @code{ishell}, and defaults to the value of the environment
variable @code{SHELL}.
@code{:gdb program} starts up gdb in the current window on
the specified program.  For both of these, you are automatically placed
in insert mode, where you should be able to interact as you would expect
to, except that @key{ESC} will take you into command mode.  While in command
mode, hitting return will send the current line as input to the
shell/gdb, similar to command-line editing vi-style in bash, but will
leave you in command mode. @refill

@node Extended marks,Register enhancements,Shell window,Enhancements
@subsection Extended marks

The marks used by the mark command @kbd{m} are emacs markers, thus they mark
a position in a buffer, not necessarily the current one.  This affects
the goto mark commands @kbd{`} and @kbd{'}.
For example, if mark @kbd{a} is placed
in one buffer, then later in another buffer, the command @kbd{`a} is typed,
evi will first switch to that buffer, then go to the location in that
buffer.  @kbd{'} and @kbd{`} also accept @kbd{.} and @kbd{,} for pop context,
and unpop context respectively.
Thus, @kbd{'.} will take you to the previous context
(defined as in vi by a region of relative motion, with an `absolute'
motion pushing a new context.  quotes surround `absolute' because a
search is considered an absolute motion for this purpose), and @kbd{'.}
will take you to the context before that.  There is a ring of 10
contexts so after 10 @kbd{'.} commands you'll end up at the original
previous context.  `Unpop context' means move forward thru the ring.
@kbd{''} and @kbd{``} are defined as exchange current
location with the location
of the previous context.  The context ring is buffer local, so use of it
will always keep you in the same buffer. @refill

@node Register enhancements,Language/mode specific editing,Extended marks,Enhancements
@subsection Register enhancements

Two changes involving registers.  First, @kbd{['}, and @kbd{["}, are new
commands which allow you to insert literal text directly into a
register.  @kbd{['} inserts a single character, and @kbd{["} inserts a string.
E.g.  @kbd{["hello@key{ESC}} inserts the string @samp{hello} into the unnamed
register, and @kbd{"a['/} inserts a slash into register a.  Second, the
register specification @kbd{"^} specifies appending to the unnamed register
(the one that gets used when no register is specified).  E.g.,
@kbd{"^["ick@key{ESC}} appends @samp{ick} to the unnamed register. @refill

@node Language/mode specific editing,,Register enhancements,Enhancements
@subsection Language/mode specific editing

@kbd{%} exhibits language sensitivity in that it ignores parentheses
embedded in quotes.  What defines quotes is based on what minor mode
emacs is in (such as c-mode or lisp-mode), or you can roll your own (see
emacs command modify-syntax-entry).

@kbd{=} is no longer specific to @code{:set lisp}.  It indents according to the
mode.  See emacs command indent-according-to-mode.

@node New operators,Differences,Enhancements,Top
@section New operators

The @kbd{*} operator can be used to send text to emacs processes.  @kbd{*}
prompts for the name of a process buffer, and the region specified is
sent to that process.  Subsequent invocations of @kbd{*} will use the same
process buffer as last specified as a default.  E.g., to send the
current line of text as a command to the emacs shell (see @code{:shell}),
type @kbd{***shell*@key{RET}}, or if the shell is already the default, just
@kbd{**@key{RET}}.  Paragraph motion or parenthesis match is often perfect for
sending function definitions to an interpreter, e.g. place the cursor at
the beginning of the function,
and type @kbd{*@}@key{RET}} or @kbd{*%@key{RET}}.  If the
function def is less easily described you can use @kbd{m.} and @kbd{yR}
described above.  In specifiying the process buffer, you can use buffer
completion using space or tab. @refill

I'm experimenting with some new complex operators.  I'm particularly
interested in your thoughts on these:

@kbd{[@{} operates over lines in a region.  It takes a motion, and a sequence
of operations to perform on each line in the region defined by the
motion.  The sequence of operations is prompted for on the bottom line.
Double the @kbd{@{} to operate on whole lines.  The point starts in the first
column for each line operated on.  For example:
@example
[@{@}i> @kbd{C-v}@key{ESC}@key{RET}
@end example
would prefix every line in the rest of the current paragraph with @samp{> }.
The @kbd{C-v @key{ESC}} sequence inserts an @key{ESC} into the string you are
entering so that it will terminate input when the loop body is executed,
not as you are entering the command.  For example:
@example
10[@{@{i/* @kbd{C-v}@key{ESC}A */@kbd{C-v}@key{ESC}@key{RET}
@end example
would place C-style comments around the next 10 lines.

@kbd{[(} defines a parameterized macro body.  A parameterized macro is diff-
erent from standard macro text in that it is parameterized by prefix
count and register specification.  In the body of such a macro, there
are two special commands: @kbd{#} and @kbd{&}.
@kbd{#} is replaced by the prefix
count applied to this macro, and @kbd{&} is replaced by the register
specification applied to this macro.  For example: @refill
@example
"a8[(j#w&dw@key{RET}
@end example
would go down one line, move over 8 words, then delete the next word
into register `a'.  This is rather contrived, but it gives you the idea.
Param- eterized macro bodies are obviously not very useful typed out
each time, and are intended to be the body of a map macro.  For example:
@example
:map M [(j#w&dw\e@key{RET}
"a8M
@end example
would be a much more likely scenario for the use of such a macro.

@node Differences,Ex commands,New operators,Top
@section Differences

The following vi commands behave differently in Evi or not implemented:

@table @kbd
@item C-@@
(insert mode only) Not implemented.
@item C-t
(pop tagstack) Not implemented.
@item #
(as function key prefix for @code{:map}) Not implemented.
@item Q
Quits evi mode, returning you to emacs, similiar to the way @kbd{Q}
in vi quits visual mode, returning you to ex.
@end table

Digit registers don't work entirely correctly - there are circumstances
in which separate lines of a change/deletion are supposed to go into
separate registers

@code{:set lisp} has no effect, however, emacs does largely take care of
any lisp'ish behaviour you'd want automatically if the file you're
editing is suffixed with @file{.l} or @file{.el}.
One particular loss, however,
is that @kbd{)} and @kbd{(} don't work on s-expressions
like they would in vi with lisp set. @refill

In vi, @code{:k}, does exactly what @code{:mark} does.  In Evi, @code{:k},
which is short for @code{:kill}, instead kills the current buffer.  See
section on File and Buffer management. @refill

@node Ex commands,Note to vip users,Differences,Top
@section Supported ex commands and variable settings

The following ex commands are supported in Evi:
@example
abbrev, cd, chdir, copy, delete, edit, file, global, map, mark, move, next,
preserve, print, put, quit, read, recover, set, shell, source, substitute,
tag, unabbrev, unmap, write, wq, yank, !, <, >, &
@end example

The following ex variable settings are supported in Evi:
@example
autoindent, errorbells, ignorecase, magic, readonly, scroll, shell,
shiftwidth, showmatch, tabstop, timeout, wrapmargin, wrapscan
@end example

@node Note to vip users,Bugs,Ex commands,Top
@section Note to vip users

Undo does not continue via @kbd{.}.  This is incompatible with vi - the
sequence @kbd{u.} in vi means `undo, then do again', whereas in vip it means
`undo, then undo some more.'  For the vip functionality use
evi-undo-more, described in the section on enhancements.

The vip commands for editing a file (@kbd{v} and @kbd{V}) and switching buffers
(@kbd{s}, and @kbd{S}) are not supported.
Use @code{:e}, @code{:E}, @code{:b}, and @code{:B}
instead.  See previous section on file and buffer management, or try
these cute macros which are (mostly) functional replacements, including
doing file and buffer completion (note the space on the end of the
line): @refill
@example
:map v :edit |map V :Edit |map K :kill |map s :buffer |map S :Buffer 
@end example

@code{:q} exits emacs.  I believe the default behaviour in vip is to simply
kill the current buffer (a concept vi doesn't really have) - either that
or quit vi emulation.  Any of these choices is reasonable, however,
given @code{:k} for killing buffers (a new command for a new concept), and
@kbd{Q} for exiting vi emulation, I chose to have @code{:q} do exactly what it
does in vi.

@node Bugs,Mailing list,Note to vip users,Top
@section What to do if things go wrong

If you encounter problems using evi (bugs, glitches, unnecessarily annoying
`features', etc), please email me a description of the problem, in as much
detail as you can.  If possible, use the @code{:bug} command in evi, which
places you in a mail buffer so you can compose your report.  It takes as
argument a decription of the problem (to be used as the subject line).  If
you can't use @code{:bug}, my address is jlewis@@cse.ogi.edu.  Please supply
the version number of evi and the version of emacs you are using.  Evi's
version number can be found via @code{:version}, or by looking at the top
of @file{evi.el}.  In the meantime, if Evi doesn't seem to be responding, or
you're having difficulty getting out of some mode, first try @kbd{C-c} (this
is vi's interrupt character), then if you're still in a pickle, try @kbd{C-g}
(this is emacs' interrupt character - there are some situations that you might
get into where it works instead of @kbd{C-c}).

@node Mailing list,Getting a copy of Evi,Bugs,Top
@section Mailing list

There is a mailing list for discussion of evi - usage, bugs, new features,
etc.  To join the list, send a note to evi-list-request@@brandx.rain.com
including your e-mail address in the body of the message.  To submit to
the list, send mail to evi-list@@brandx.rain.com, or use the @code{:evilist}
command.

@node Getting a copy of Evi,Credits,Mailing list,Top
@section Getting a copy of Evi

You can obtain a copy of Evi via FTP from an elisp archive site, or from
the mail-server on my machine (which will assure the most up-to-date
version).  The main elisp-archive is at archive.cis.ohio-state.edu (in
the directory gnu/emacs/elisp-archive) - Evi is in modes/evi.el.Z.  To
get it from my mail-server, send a message to
mail-server@@brandx.rain.com containing:
@example
send evi.el
@end example
-- or --
@example
send evi.el.Z
@end example
followed by:
@example
end
@end example
@code{evi.el.Z} will be uuencoded for delivery.  Also, add `send index'
if you want the file sizes.  You can also retrieve the source for this
document in either evi.tex or evi.info.

@node Credits,,Getting a copy of Evi,Top
@section Credits

Masahiko Sato - for having the audacity to write vip in the first place ;-)
Eric Benson - for being the first real advocate
and those helpful folks who gave feedback on the first release
and the following immensely helpful souls: James Montebello, Steven Dick,
Mike Sangrey, Brian Divine, Bill Reynolds, Roger E. Benz, Richard Ryan,
Volker Englisch, John Haugen, Calvin Clark, Sam Falkner, Mark Pulver,
Cameron Gregory, David Muir Sharnoff and last, but hardly least, Jamie
Zawinski.

@bye
